\section{Capstone Design Requirements}
Our project will fulfill the Robotics capstone design requirement by touching on all of the major cornerstones of robotic technology. Mechanical Engineering will be fulfilled through designing a new link for the 3001 arm and developing the software required to control that link. We will need to use methods such as DH parameters, inverse and forward kinematics, and motor torque requirements to ensure the mechanical ability of this arm and control it. Dynamic control algorithms will be explored. We will also need to use computer programs such as ROS and Code Composer Studio in order to control the arm. Additionally we will design a software package and create an interactive GUI so that end-users are able to easily interact with the arm. We will need to write lower level embedded code in order to control the electronics on the master controller and each link's microcontroller. With all of the electrical component design in addition to the communications protocols we cover most topics in microelectronic circuit design and implementations. Finally, Systems engineering will be used for the designing the electrical, mechanical and computer science portions of the project. We will need to use systems engineering skills that we have learned through our various robotics projects in order to complete the project.
\subsection{Computer Science}
In order to properly communicate with the arm, we will need to implement communication protocols. The communications will need to be efficient. The problem of sending information to each of the links of the Robotic Arm will rely heavily on Information Theory.\\
\newline
In addition to communications, we will be developing a set of programs to interface with the arm. The suite will include a library to offload the computationally heavy task of solving kinematic equations to direct the motion of the arm. It will also include an application so the user can interact with the functions inside the mathematics library. This application will have a GUI so the user can specify the arm's configuration. It will also provide the user with some basic control functions, such as jogging the arm's position (either joint-by-joint or linearly with respect to the base coordinate frame) and opening/closing the gripper. Another feature will be the ability to record poses for the arm to go to.\\
\newline
As the arm is designed with modularity in mind, the software will need to be dynamic enough to leave the user with plenty of room for configuration. One of the deliverables of this project will be a set of files that can be included in other people's work so that anybody has access to code that they can use to control the robotic arm.\\
\newline
Finally, we'll need to write firmware that will run on each of the arm's joints, and libraries that people will be able to use to interact with those joints. 

\subsection{Electrical and Computer Engineering}
The design and implementation of each joint's control board involves core topics from Electrical Engineering. The controller board designed for each link must be carefully designed and tested rigorously. Electrical systems design skills will be used to select compatible parts to produce the desired functionality. \\
\newline
Once constructed, the controller board will need embedded code to control the motor. This code will need to be tested and optimized for better performance. Some communications protocols will be used to send data from chip to chip and board to board.



