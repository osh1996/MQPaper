\section{Methodology}
\subsection{Kit Components}
We decided to break a robot arm down into its component parts. We came up with the main parts of our kit: sticks, joints, a base, and end of arm tools. This breakdown was to try to maximize modularity while keeping the pieces relatively simple. By separating the joint from the stick, we can have multiple sticks, which are easy to manufacture, of different types and lengths to offset a few types of joints, which are difficult to manufacture. The base is necessary to send and receive computation control from a computer. Multiple end of arm tools are needed to provide functionality to the arm besides movement.

\subsection{Connectors}
The connectors are vitally important to the functionality of this project. A good connector will need to make solid mechanical and electrical connection between parts while also providing the ability to quickly connect and disconnect parts. In addition, the type of connection we choose will affect the modularity of the system as a whole. The important things to note while deciding on criteria for the connector are how/if they will be keyed, how they will pass electrical signals, where exactly the connector will be on the joints, and how the connector will be secured.

\subsubsection{Connector Position on Joints}
Connector position is the first major design choice we had to make. They can be positioned either on the axis of rotation of the joint or off the axis of rotation of the joint, and selecting one method versus the other vastly changes the way that the connector would work. Putting the connector ON the axis of rotation means that the connector would connect to the joint axis directly, while putting it OFF the axis of rotation means the connector would connect to a piece that is connected to the joint axis. \\
\newline
The advantage that placing the connector off the axis of rotation has over placing the connector on the axis of rotation is that the joint will be a solid unit. Having the joint be a solid unit seems a better design choice than splitting it in half, so we decided to go with putting connectors off the axis of rotation.

\subsubsection{Securing the Connection}
%available options: no tools, single screw, multiple screws, slots with removable pins
One of the important aspects of a modular system is how easy it is to connect or disconnect parts to or from that system. The main options for quick connections are requiring no additional hardware, requiring a single screw, and requiring slots and pins. \\
\newline
The most obvious solution is to connect parts with no additional hardware required. This creates a very complicated design challenge since using no tools means the user would have to secure any connection with just their hands. This can weaken the joint mechanically. The advantage to this method has is that it is fairly quick. \\ 
\newline
A step down in the simplicity solution is to require a single screw to join two pieces together. This is still simple and pieces can be connected somewhat quickly, but does require a tool to connect pieces. The main advantage is that screws hold parts together very well and the mechanical integrity of the connection should be held. \\
\newline
Another option is to design the joints and sticks in such a way that they slot together and are held in place with pins. This requires additional hardware, but no tools. This should keep pieces together fairly well while still allowing connections to be made quickly and easily. \\
\newline


\subsubsection{Keying the Connector}
Keying the mechanical connection between joints, sticks, and the base changes how modular the system is overall as well as how many unique components will be needed in each kit. Not keying the connection is not an option since this would allow the user to connect the pieces together in any orientation and the orientation needs to be known in order to accurately control the arm. This leaves two main options for the connections: keying for 1 orientation and keying for 4 orientations.\\
\newline
Keying the connectors for 1 orientation means that three different kinds of revolute joint must be design to fully represent the ways a revolute joint can move in 3D space. Essentially, this would mean each different joint would rotate about a different axis relative to the connector axis. The modularity of the kit is impacted quite negatively by doing this, since each joint can only connect in one way and therefore cannot be used where another type of rotation is needed. This design is quite simple, however, since the connectors don't need to be rotationally symmetric about any axis. \\
\newline
Keying the connectors for 4 orientations presents a slightly more challenging design problem, however. The connectors would need to be evenly rotationally symmetric 4 ways about the axis of connection in order for this design to work. 4-way keyed connectors will bring the number of unique joints down from 3 to 2. Doing this does help with the modularity of the design, though, since the rotational joint can be implemented to rotate in either axis perpendicular to the connector axis. A disadvantage of this configuration is an increase in complexity. \\
\newline
Since additional complexity when designing is less important that the overall modularity, we decided to go with a 4-position keyed connection. This allows a single rotational joint and only 1 right angle stick design so that the user can construct many different kinds of arms from these simple parts.

\subsubsection{Passing Signals}
% rigid connectors on joints, wires running along outside, wires running through inside that pop out at connector
Connectors also need to pass the power and signal buses through from joint to stick or joint. This can be done in one of a few ways, including: rigid mechanical connectors, loose wires running along the outside of parts, wires running inside of parts, and wires connecting internal bus bars. \\
\newline
Rigid mechanical connectors for passing signals would make connection when parts are connected together. These connections would have to be evenly rotationally symmetric 4 ways about the axis of rotation since the mechanical connectors are. A disadvantage of these connectors is that they rely on the integrity of the mechanical connection to pass electrical signals properly. If the connection flexes or bends too much then the electrical connection could break even though the mechanical connection is still mostly intact. Another disadvantage is that this is the most costly option for passing electrical signals, requiring 4 connectors per connection. \\
\newline
Loose wires along the outside of the parts have several advantages over rigid mechanical connectors. The first of which is that they only require one set of connections per connector. This reduces the cost of each connector significantly. The main disadvantage of this kind of electrical connection is that the wires could get snagged on something since the system is supposed to be active and moving. Another disadvantage is that the wires need to have enough slack to move with the arm without limiting the arm's movement.  \\
\newline
Wires running through the parts that pop out at the connectors is another option or passing signals along the system. This option is practically the same cost as external wires, but doesn't have the problem of wires snagging on the environment. Unfortunately this doesn't solve the problem of wires needing lots of slack to allow for movement of the whole system. \\
\newline
Short wires that connect some internal bus bars provide a more expensive solution to this problem. This would remove the problem of wires needing slack for the entire system. Instead, wires would only have enough slack for one joint. Doing this does bring some complexity issues, however, since the bars would have to be designed into the system and not added on at the end. \\
\newline
For our design we decided to use internal wires running the length of the system. The low cost and simplicity of this solution outweighs the negatives of having to add lots of extra wire to account for movement of the system.\\

\subsection{Sticks}
%things to decide for sticks: material, lengths, types (straight, right angle)

Sticks are the things that connect joints together and space joints out. They do not have any electronics on board; they simply pass power and communication wires along to the rest of the arm. They need to be strong, light weight, and cheap.\\
\newline
Sticks have an input side and an output side. Two kinds of sticks will need to be created: one will be straight, and one will have a right angle at the input side.

\pagebreak

%yes I like this
%But it makes the page before look so awkward and sad. And it still puts the footnotes too high up


%The following defines two counter variables: startNum and runCount. Every time a new footnote is declared, we increment runCount by 1. Then, when we're populating the footnotes with words, we start with number startNum and count upwards.
\FPeval{\startNum}{(thefootnote)}
\FPeval{\startNum}{round((\startNum) + 1,0)}
\FPeval{\runCount}{\startNum}

\begin{table}[H]
	\begin{center}
		\begin{tabular}{ | l | l | p{1.5cm} | p{2.5cm} | p{2.5cm}| l |}
			\hline
			Stick Material & Cost/kg & Cost/ 20mm & Rigidity & Complexity & Weight \\ \hline
			3D-printed PLA\footnotemark[\runCount] & ~\$20/kg & \$3 & Might break  & Low: Very few constraints on possible designs & 150g \footnotemark[\runCount]
			\\ \hline
			\FPeval{\runCount}{round((\runCount)+1,0)}%
			
			PVC\footnotemark[\runCount] & \$7 & \$0.70 & Bends over time & Connect/ Disconnect easily & 106g\\ 
			\hline
			%I would have thought that the next line needed to add 1, not 2. My working hypothesis is that global variables can't really be modified from inside a table. I don't get it. Clearly the line is doing something, or else the \footnotemark line would simply do nothing.
			\FPeval{\runCount}{round((\runCount)+2,0)}%
			
			Carbon Fiber\footnotemark[\runCount] & \$66 & \$6 & Strong, but possibly too thin & & 58g\\ 
			\hline
			%I'm surprised that the next line adds 3 instead of 1. It works this way though. Oh well.
			\FPeval{\runCount}{round((\runCount)+3,0)}%
			
			80/20\footnotemark[\runCount] & \$2.46 & \$2 & Not going anywhere & Nice connecting options & 154g\\ 
			\hline
			%Only needed if more footnotes are added
			%\FPeval{\runCount}{round((\runCount)+1,0)}%
			
		\end{tabular}
		
	\end{center}
	\caption{Comparison of materials to construct sticks}
	\label{tbl:stick_materials}
\end{table}
\footnotetext[\startNum]{This number assumes 100\% infill. The actual number will almost certainly be lower.}
\FPeval{\startNum}{round((\startNum)+1,0)}

\footnotetext[\startNum]{\url{http://www.homedepot.com/p/Formufit-1-in-x-5-ft-Furniture-Grade-Sch-40-PVC-Pipe-in-White-P001FGP-WH-5/205171542?cm\_mmc=Shopping\%7cTHD\%7cG\%7c0\%7cG-BASE-PLA-D26P-Plumbing\%7c&gclid=Cj0KCQjwx8fOBRD7ARIsAPVq-Nlw\_xbuCOf-QHORvUW4gQ4Dx7SiZt\_vqQ3OvxBdTW-eckQhdp5WWFYaAs9DEALw\_wcB&gclsrc=aw.ds&dclid=CIagtKLB09YCFUuraQodN\_MAOQ}}

\FPeval{\startNum}{round((\startNum)+1,0)}

\footnotetext[\startNum]{\url{https://www.rockwestcomposites.com/45552?gclid=Cj0KCQjwx8fOBRD7ARIsAPVq-NmCNUg6ULxgd9udG-xSuPtJuHKgCLjUSgX\_zXPDgRr2CmKU0tSXX-waAgb9EALw\_wcB}}

\FPeval{\startNum}{round((\startNum)+1,0)}

\footnotetext[\startNum]{\url{https://8020.net/1010.html}}



We choose to use 3D-printed PLA. While it's not the absolute cheapest option, nor is it the lightest one, its high configurability makes it the ideal material for our needs - especially given its availability for potential customers; anybody with a 3D-printer would be able to make one of our arms. Also, while aesthetic concerns should not be the only factor, we're allowed to consider the way the final product would look. An arm made from PVC would reflect poorly on all the involved parties.


\subsection{Motor Selection}
\begin{table}[H]
	\begin{figure}[H]
		\includegraphics[width=\textwidth]{Pictures/motor_chart}
	\end{figure}
	\caption{Comparison of Possible Motors}
	\label{tbl:Motor_Chart}
\end{table}


To choose our motors, we looked for high-torque, low-cost DC motors. We chose to go with DC Brushed motors to control our arm because they are quiet, low-cost, vibration free and fairly efficient. We also considered using Dc Brushless motors as well as Stepper Motors, but each had their own pros and cons. Brushless motors cost much more than comparable brushed motors, and require complicated control logic to operate. Stepper motors were a good option due to their ability to be backdriven and their built in discrete steps for controlling. But, they do not operate well under conditions where the load changes significantly in a short period of time and also require external control to keep track of the position.  \\
\newline
Once we decided to use brushed DC motors, our next step was to find suitable motors that fit our criteria of high torque and low cost. We found 2 categories of motors that seemed to fill these requirements, planetary gearbox motors and spur gearbox motors. Planetary gearboxes work by having multiple "planet" gears revolving around a central "sun" gear that rotates in place. They are named for their resemblance of the planets orbiting around the sun. All of the "planet" gears are held in place by an outer "ring" gear that acts to keep the "planet" gears in contact with both the "sun" and the "ring". By having these idler gears rotating around a central axis, you can have torque transferred linearly, without the need for offset shafts, greatly reducing the total size of the gearbox. With multiple gears transferring the torque load at one time, the individual load on each tooth is lowered making these perfect for high torque applications. Spur gearboxes on the other hand use linear offset shafts that transfer the entire torque from one gear to the next until the output shaft in a direct chain.  This means that they wear out much faster since the torque load is much higher on individual gears and teeth.  Therefore, since we need a reliable high torque motor, we decided to go with planetary gear motors.  \\
\newline
Once we had made the decision to go with a planetary gearbox brushed DC motor, we made a chart as seen in Table 5 of possible motors that had high stall torques.  One final decision that we had to make was whether or not to purchase a motor with a rotary shaft encoder. Rotary shaft encoders provide easy control over DC motors by relaying the position of the shaft before the gearbox on the motor.  This allows for high resolution control in the case of high gear reductions but also costs a fair amount extra to purchase with the motor.  Considering that the motor is just a part of the joint and we care more about the position of the overall joint rather than the motor itself, we decided to save the money and go with a cheaper non-encoder motor.  By doing this we are moving the point at which we control the joint system from the motor to the joint if we use an absolute encoder on the joint shaft.  This results in a closed loop control system which is optimal for our situation and cost-effective.  

\subsection{Control Board Part selection}
Selecting the types of sensors to use for the control board was a very important step of the control board design. There are many different types of sensors to accomplish each major goal that the control board must accomplish.
\subsubsection{Joint Angle Sensor}
Potentiometers seem like a good choice due to their simplicity and high accuracy capabilities. However, they do not lend themselves well to this application because of how quickly they wear out. Over time, as the joints move to different positions, the potentiometers will wear out quickly and cause inaccurate readings. Additionally, long lifespan and high resolution potentiometers can be very expensive. Furthermore, potentiometers are large and can be difficult to mount. Finally, the hard stop on the potentiometer means the joint angles will be limited to a certain range (typically about 270 \textdegree  for single turn potentiometers).\\
\newline
The next obvious solution is to use optical encoders because they will not wear out and offer very high resolution capabilities. These sensors are not well suited for this application, however, since they are typically expensive, especially for high resolution encoders - and ones that are capable of reading absolute position. Additionally these sensors are somewhat bulky and would take up too much space in the closed environment of a joint. \\
\newline
This leaves us with hall effect sensors. These sensors are very small and moderately high resolution while also being a contact-free sensor, so wearing them out will not be a concern. A main concern with hall effect sensors is that they need to be mounted somewhat precisely and carefully. Traditional machining methods make this difficult to accomplish, but 3D printing allows us to easily overcome this challenge.  Another concern is external electromagnetic interference, but with somewhat careful circuit board design, we should be able to minimize this issue.

\subsubsection{Motor Current Sensor}
A shunt resistor seems practical due to the simplicity of the design, but careful designing must be done in order to get the noise levels down to a reasonable amount. In addition to this, the power loss when using a shunt resistor could cause the arm to stall before anticipated. When the shunt resistor takes power from the motor, the whole motor curve slides inward, decreasing the maximum power output. Trace resistance would be a good alternative, but requires calibration after the circuit is constructed. \\
\newline
Instead of these, we decided to use a hall effect current sensor. Hall effect current sensors are ready-made sensors that give low noise, properly calibrated outputs, are not very expensive, and are easy to integrate into a circuit design. These sensors have extremely small power losses to the motor. The main drawback of these sensors is that they have a low bandwidth, but we are using DC motors so this should not be a problem. Some care will need to be taken when placing these on the circuit, however, since they are sensitive to external magnetic fields.

\subsubsection{Off-board Communication}
SPI and I$^2$C are mostly used for on-board, controller-to-peripheral communications and therefore are not a good choice for the base to control board communication. RS232 is not a good solution for this problem either because it is a single transmitter and single receiver per line. This leaves RS485 and CAN. \\
\newline
RS485 and CAN are similar in many ways, but with a few key differences that separate them. RS485 is very fast to transmit and simple to implement, but takes a lot of the controller's time to send packets. CAN has the advantage because the controller and transceiver control the transmission independent of the controller so the controller has more free time to process data. Another advantage CAN has over RS485 is the amount of error checking that goes on to ensure proper message transmission. For these reasons, we decided to use CAN to communicate between the base and control boards.
% TYPES: SPI, I2C, RS232, RS485, CAN

\subsection{Arm Structure}
Arm structure is not something we wanted to define, since the end user is supposed to create their own arms, but there were some basic things we needed to define. The first of these is that every arm must begin with a base and end with an end effector. This is because the central CAN bus must be terminated with resistors at both ends. An alternative to this is to have every piece terminate the CAN bus if it is the last piece in the chain, but this creates unnecessary complexity in each piece. The second constraint placed on arm construction is that sticks cannot connect to other sticks. This is because we didn't want the user to construct an arm with ridiculous length that would be impossible to lift.

\subsection{Arm Base}
% Talk about requirements for base
% take in new joint data at certain speeds
% output new joint data at certain speeds

\subsection{End-of-Arm Tooling}







